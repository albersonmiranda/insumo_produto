% Options for packages loaded elsewhere
\PassOptionsToPackage{unicode}{hyperref}
\PassOptionsToPackage{hyphens}{url}
\PassOptionsToPackage{dvipsnames,svgnames,x11names}{xcolor}
%
\documentclass[
  letterpaper,
  DIV=11,
  numbers=noendperiod]{scrreprt}

\usepackage{amsmath,amssymb}
\usepackage{iftex}
\ifPDFTeX
  \usepackage[T1]{fontenc}
  \usepackage[utf8]{inputenc}
  \usepackage{textcomp} % provide euro and other symbols
\else % if luatex or xetex
  \usepackage{unicode-math}
  \defaultfontfeatures{Scale=MatchLowercase}
  \defaultfontfeatures[\rmfamily]{Ligatures=TeX,Scale=1}
\fi
\usepackage{lmodern}
\ifPDFTeX\else  
    % xetex/luatex font selection
  \setsansfont[]{Times New Roman}
  \setmonofont[Scale=0.8]{Fira Code}
\fi
% Use upquote if available, for straight quotes in verbatim environments
\IfFileExists{upquote.sty}{\usepackage{upquote}}{}
\IfFileExists{microtype.sty}{% use microtype if available
  \usepackage[]{microtype}
  \UseMicrotypeSet[protrusion]{basicmath} % disable protrusion for tt fonts
}{}
\makeatletter
\@ifundefined{KOMAClassName}{% if non-KOMA class
  \IfFileExists{parskip.sty}{%
    \usepackage{parskip}
  }{% else
    \setlength{\parindent}{0pt}
    \setlength{\parskip}{6pt plus 2pt minus 1pt}}
}{% if KOMA class
  \KOMAoptions{parskip=half}}
\makeatother
\usepackage{xcolor}
\setlength{\emergencystretch}{3em} % prevent overfull lines
\setcounter{secnumdepth}{5}
% Make \paragraph and \subparagraph free-standing
\ifx\paragraph\undefined\else
  \let\oldparagraph\paragraph
  \renewcommand{\paragraph}[1]{\oldparagraph{#1}\mbox{}}
\fi
\ifx\subparagraph\undefined\else
  \let\oldsubparagraph\subparagraph
  \renewcommand{\subparagraph}[1]{\oldsubparagraph{#1}\mbox{}}
\fi

\usepackage{color}
\usepackage{fancyvrb}
\newcommand{\VerbBar}{|}
\newcommand{\VERB}{\Verb[commandchars=\\\{\}]}
\DefineVerbatimEnvironment{Highlighting}{Verbatim}{commandchars=\\\{\}}
% Add ',fontsize=\small' for more characters per line
\usepackage{framed}
\definecolor{shadecolor}{RGB}{48,48,48}
\newenvironment{Shaded}{\begin{snugshade}}{\end{snugshade}}
\newcommand{\AlertTok}[1]{\textcolor[rgb]{1.00,0.81,0.69}{#1}}
\newcommand{\AnnotationTok}[1]{\textcolor[rgb]{0.50,0.62,0.50}{\textbf{#1}}}
\newcommand{\AttributeTok}[1]{\textcolor[rgb]{0.80,0.80,0.80}{#1}}
\newcommand{\BaseNTok}[1]{\textcolor[rgb]{0.86,0.64,0.64}{#1}}
\newcommand{\BuiltInTok}[1]{\textcolor[rgb]{0.80,0.80,0.80}{#1}}
\newcommand{\CharTok}[1]{\textcolor[rgb]{0.86,0.64,0.64}{#1}}
\newcommand{\CommentTok}[1]{\textcolor[rgb]{0.50,0.62,0.50}{#1}}
\newcommand{\CommentVarTok}[1]{\textcolor[rgb]{0.50,0.62,0.50}{\textbf{#1}}}
\newcommand{\ConstantTok}[1]{\textcolor[rgb]{0.86,0.64,0.64}{\textbf{#1}}}
\newcommand{\ControlFlowTok}[1]{\textcolor[rgb]{0.94,0.87,0.69}{#1}}
\newcommand{\DataTypeTok}[1]{\textcolor[rgb]{0.87,0.87,0.75}{#1}}
\newcommand{\DecValTok}[1]{\textcolor[rgb]{0.86,0.86,0.80}{#1}}
\newcommand{\DocumentationTok}[1]{\textcolor[rgb]{0.50,0.62,0.50}{#1}}
\newcommand{\ErrorTok}[1]{\textcolor[rgb]{0.76,0.75,0.62}{#1}}
\newcommand{\ExtensionTok}[1]{\textcolor[rgb]{0.80,0.80,0.80}{#1}}
\newcommand{\FloatTok}[1]{\textcolor[rgb]{0.75,0.75,0.82}{#1}}
\newcommand{\FunctionTok}[1]{\textcolor[rgb]{0.94,0.94,0.56}{#1}}
\newcommand{\ImportTok}[1]{\textcolor[rgb]{0.80,0.80,0.80}{#1}}
\newcommand{\InformationTok}[1]{\textcolor[rgb]{0.50,0.62,0.50}{\textbf{#1}}}
\newcommand{\KeywordTok}[1]{\textcolor[rgb]{0.94,0.87,0.69}{#1}}
\newcommand{\NormalTok}[1]{\textcolor[rgb]{0.80,0.80,0.80}{#1}}
\newcommand{\OperatorTok}[1]{\textcolor[rgb]{0.94,0.94,0.82}{#1}}
\newcommand{\OtherTok}[1]{\textcolor[rgb]{0.94,0.94,0.56}{#1}}
\newcommand{\PreprocessorTok}[1]{\textcolor[rgb]{1.00,0.81,0.69}{\textbf{#1}}}
\newcommand{\RegionMarkerTok}[1]{\textcolor[rgb]{0.80,0.80,0.80}{#1}}
\newcommand{\SpecialCharTok}[1]{\textcolor[rgb]{0.86,0.64,0.64}{#1}}
\newcommand{\SpecialStringTok}[1]{\textcolor[rgb]{0.80,0.58,0.58}{#1}}
\newcommand{\StringTok}[1]{\textcolor[rgb]{0.80,0.58,0.58}{#1}}
\newcommand{\VariableTok}[1]{\textcolor[rgb]{0.80,0.80,0.80}{#1}}
\newcommand{\VerbatimStringTok}[1]{\textcolor[rgb]{0.80,0.58,0.58}{#1}}
\newcommand{\WarningTok}[1]{\textcolor[rgb]{0.50,0.62,0.50}{\textbf{#1}}}

\providecommand{\tightlist}{%
  \setlength{\itemsep}{0pt}\setlength{\parskip}{0pt}}\usepackage{longtable,booktabs,array}
\usepackage{calc} % for calculating minipage widths
% Correct order of tables after \paragraph or \subparagraph
\usepackage{etoolbox}
\makeatletter
\patchcmd\longtable{\par}{\if@noskipsec\mbox{}\fi\par}{}{}
\makeatother
% Allow footnotes in longtable head/foot
\IfFileExists{footnotehyper.sty}{\usepackage{footnotehyper}}{\usepackage{footnote}}
\makesavenoteenv{longtable}
\usepackage{graphicx}
\makeatletter
\def\maxwidth{\ifdim\Gin@nat@width>\linewidth\linewidth\else\Gin@nat@width\fi}
\def\maxheight{\ifdim\Gin@nat@height>\textheight\textheight\else\Gin@nat@height\fi}
\makeatother
% Scale images if necessary, so that they will not overflow the page
% margins by default, and it is still possible to overwrite the defaults
% using explicit options in \includegraphics[width, height, ...]{}
\setkeys{Gin}{width=\maxwidth,height=\maxheight,keepaspectratio}
% Set default figure placement to htbp
\makeatletter
\def\fps@figure{htbp}
\makeatother

\KOMAoption{captions}{tableheading}
\renewcommand\thesubsection{\alph{subsection}}
\makeatletter
\@ifpackageloaded{caption}{}{\usepackage{caption}}
\AtBeginDocument{%
\ifdefined\contentsname
  \renewcommand*\contentsname{Índice}
\else
  \newcommand\contentsname{Índice}
\fi
\ifdefined\listfigurename
  \renewcommand*\listfigurename{Lista de Figuras}
\else
  \newcommand\listfigurename{Lista de Figuras}
\fi
\ifdefined\listtablename
  \renewcommand*\listtablename{Lista de Tabelas}
\else
  \newcommand\listtablename{Lista de Tabelas}
\fi
\ifdefined\figurename
  \renewcommand*\figurename{Figura}
\else
  \newcommand\figurename{Figura}
\fi
\ifdefined\tablename
  \renewcommand*\tablename{Tabela}
\else
  \newcommand\tablename{Tabela}
\fi
}
\@ifpackageloaded{float}{}{\usepackage{float}}
\floatstyle{ruled}
\@ifundefined{c@chapter}{\newfloat{codelisting}{h}{lop}}{\newfloat{codelisting}{h}{lop}[chapter]}
\floatname{codelisting}{Listagem}
\newcommand*\listoflistings{\listof{codelisting}{Lista de Listagens}}
\makeatother
\makeatletter
\makeatother
\makeatletter
\@ifpackageloaded{caption}{}{\usepackage{caption}}
\@ifpackageloaded{subcaption}{}{\usepackage{subcaption}}
\makeatother
\ifLuaTeX
\usepackage[bidi=basic]{babel}
\else
\usepackage[bidi=default]{babel}
\fi
\babelprovide[main,import]{brazilian}
% get rid of language-specific shorthands (see #6817):
\let\LanguageShortHands\languageshorthands
\def\languageshorthands#1{}
\ifLuaTeX
  \usepackage{selnolig}  % disable illegal ligatures
\fi
\usepackage{bookmark}

\IfFileExists{xurl.sty}{\usepackage{xurl}}{} % add URL line breaks if available
\urlstyle{same} % disable monospaced font for URLs
\hypersetup{
  pdftitle={Insumo-Produto},
  pdfauthor={Alberson da Silva Miranda},
  pdflang={pt-BR},
  colorlinks=true,
  linkcolor={blue},
  filecolor={Maroon},
  citecolor={Blue},
  urlcolor={Blue},
  pdfcreator={LaTeX via pandoc}}

\title{Insumo-Produto\thanks{Código disponível em
https://github.com/albersonmiranda/insumo\_produto.}}
\usepackage{etoolbox}
\makeatletter
\providecommand{\subtitle}[1]{% add subtitle to \maketitle
  \apptocmd{\@title}{\par {\large #1 \par}}{}{}
}
\makeatother
\subtitle{1ª Lista de Exercícios}
\author{Alberson da Silva Miranda}
\date{23 de junho de 2024}

\begin{document}
\maketitle

\renewcommand*\contentsname{Índice}
{
\hypersetup{linkcolor=}
\setcounter{tocdepth}{2}
\tableofcontents
}
\chapter{PRIMEIRA QUESTÃO}\label{primeira-questuxe3o}

Os valores em reais das transações interindustriais e os resultados
totais para uma economia de dois setores (agricultura e manufatura) são
mostrados abaixo:

\begin{equation}\phantomsection\label{eq-q1_z}{
Z =
\begin{bmatrix}
  500 & 350 \\
  320 & 360
\end{bmatrix}
}\end{equation}

\begin{equation}\phantomsection\label{eq-q1_x}{
x =
\begin{bmatrix}
  1000 \\
  800
\end{bmatrix}
}\end{equation}

\section{Quais são os dois elementos no vetor de demanda
final?}\label{quais-suxe3o-os-dois-elementos-no-vetor-de-demanda-final}

\begin{Shaded}
\begin{Highlighting}[numbers=left,,]
\CommentTok{\# dados}
\NormalTok{data }\OtherTok{=} \FunctionTok{list}\NormalTok{(}
  \AttributeTok{Z =} \FunctionTok{matrix}\NormalTok{(}\FunctionTok{c}\NormalTok{(}\DecValTok{500}\NormalTok{, }\DecValTok{350}\NormalTok{, }\DecValTok{320}\NormalTok{, }\DecValTok{360}\NormalTok{), }\AttributeTok{nrow =} \DecValTok{2}\NormalTok{, }\AttributeTok{byrow =} \ConstantTok{TRUE}\NormalTok{),}
  \AttributeTok{x =} \FunctionTok{c}\NormalTok{(}\DecValTok{1000}\NormalTok{, }\DecValTok{800}\NormalTok{)}
\NormalTok{)}

\CommentTok{\# o vetor f da demanda final é a diferença entre Z e x}
\NormalTok{f }\OtherTok{=} \FunctionTok{with}\NormalTok{(data, x }\SpecialCharTok{{-}} \FunctionTok{rowSums}\NormalTok{(Z)) }\SpecialCharTok{|\textgreater{}}
  \FunctionTok{as.matrix}\NormalTok{(}\AttributeTok{nrow =} \DecValTok{2}\NormalTok{)}
\FunctionTok{dimnames}\NormalTok{(f) }\OtherTok{=} \FunctionTok{list}\NormalTok{(}\FunctionTok{c}\NormalTok{(}\StringTok{"setor A"}\NormalTok{, }\StringTok{"setor B"}\NormalTok{), }\FunctionTok{c}\NormalTok{(}\StringTok{"demanda final"}\NormalTok{))}
\FunctionTok{print}\NormalTok{(f)}
\end{Highlighting}
\end{Shaded}

\begin{verbatim}
        demanda final
setor A           150
setor B           120
\end{verbatim}

\section{\texorpdfstring{Suponha que \(f_1\) aumente em \(50\) e \(f_2\)
diminua em \(20\). Quais novas produções seriam necessárias para
satisfazer as novas demandas
finais?}{Suponha que f\_1 aumente em 50 e f\_2 diminua em 20. Quais novas produções seriam necessárias para satisfazer as novas demandas finais?}}\label{suponha-que-f_1-aumente-em-50-e-f_2-diminua-em-20.-quais-novas-produuxe7uxf5es-seriam-necessuxe1rias-para-satisfazer-as-novas-demandas-finais}

\begin{Shaded}
\begin{Highlighting}[numbers=left,,]
\CommentTok{\# novos valores de f}
\NormalTok{f\_new }\OtherTok{=}\NormalTok{ f }\SpecialCharTok{+} \FunctionTok{c}\NormalTok{(}\DecValTok{50}\NormalTok{, }\SpecialCharTok{{-}}\DecValTok{20}\NormalTok{)}

\CommentTok{\# matriz de coeficientes técnicos}
\NormalTok{A }\OtherTok{=} \FunctionTok{with}\NormalTok{(data, }\FunctionTok{sweep}\NormalTok{(Z, }\DecValTok{2}\NormalTok{, x, }\AttributeTok{FUN =} \StringTok{"/"}\NormalTok{))}

\CommentTok{\# matriz de Leontief}
\NormalTok{L }\OtherTok{=} \FunctionTok{diag}\NormalTok{(}\DecValTok{2}\NormalTok{) }\SpecialCharTok{{-}}\NormalTok{ A}

\CommentTok{\# inversa de Leontief}
\NormalTok{B }\OtherTok{=} \FunctionTok{solve}\NormalTok{(L)}

\CommentTok{\# nova produção final}
\NormalTok{x\_new }\OtherTok{=}\NormalTok{ B }\SpecialCharTok{\%*\%}\NormalTok{ f\_new}

\CommentTok{\# resposta}
\FunctionTok{dimnames}\NormalTok{(x\_new) }\OtherTok{=} \FunctionTok{list}\NormalTok{(}\FunctionTok{c}\NormalTok{(}\StringTok{"setor A"}\NormalTok{, }\StringTok{"setor B"}\NormalTok{), }\StringTok{"produção total"}\NormalTok{)}
\FunctionTok{print}\NormalTok{(x\_new)}
\end{Highlighting}
\end{Shaded}

\begin{verbatim}
        produção total
setor A      1138.8889
setor B       844.4444
\end{verbatim}

\chapter{SEGUNDA QUESTÃO}\label{segunda-questuxe3o}

As vendas interindústrias e a produção total em uma pequena economia de
três setores para o ano \(t\) são dadas na tabela a seguir, com valores
apresentados em milhares de reais.

\begin{Shaded}
\begin{Highlighting}[numbers=left,,]
\NormalTok{data }\OtherTok{=} \FunctionTok{list}\NormalTok{(}
  \AttributeTok{Z =} \FunctionTok{matrix}\NormalTok{(}
    \FunctionTok{c}\NormalTok{(}
      \DecValTok{350}\NormalTok{, }\DecValTok{0}\NormalTok{, }\DecValTok{0}\NormalTok{,}
      \DecValTok{50}\NormalTok{, }\DecValTok{250}\NormalTok{, }\DecValTok{150}\NormalTok{,}
      \DecValTok{200}\NormalTok{, }\DecValTok{150}\NormalTok{, }\DecValTok{550}
\NormalTok{      ),}
    \AttributeTok{nrow =} \DecValTok{3}\NormalTok{,}
    \AttributeTok{byrow =} \ConstantTok{TRUE}
\NormalTok{  ),}
  \AttributeTok{x =} \FunctionTok{c}\NormalTok{(}\DecValTok{1000}\NormalTok{, }\DecValTok{500}\NormalTok{, }\DecValTok{1000}\NormalTok{)}
\NormalTok{)}
\end{Highlighting}
\end{Shaded}

\section{Encontre a matriz de coeficientes técnicos e a matriz inversa
de
Leontief}\label{encontre-a-matriz-de-coeficientes-tuxe9cnicos-e-a-matriz-inversa-de-leontief}

\begin{Shaded}
\begin{Highlighting}[numbers=left,,]
\CommentTok{\# matriz de coeficientes técnicos}
\NormalTok{A }\OtherTok{=} \FunctionTok{with}\NormalTok{(data, }\FunctionTok{sweep}\NormalTok{(Z, }\DecValTok{2}\NormalTok{, x, }\AttributeTok{FUN =} \StringTok{"/"}\NormalTok{))}

\CommentTok{\# matriz de Leontief}
\NormalTok{L }\OtherTok{=} \FunctionTok{diag}\NormalTok{(}\DecValTok{3}\NormalTok{) }\SpecialCharTok{{-}}\NormalTok{ A}

\CommentTok{\# inversa de Leontief}
\NormalTok{B }\OtherTok{=} \FunctionTok{solve}\NormalTok{(L)}

\CommentTok{\# resposta}
\FunctionTok{print}\NormalTok{(A)}
\end{Highlighting}
\end{Shaded}

\begin{verbatim}
     [,1] [,2] [,3]
[1,] 0.35  0.0 0.00
[2,] 0.05  0.5 0.15
[3,] 0.20  0.3 0.55
\end{verbatim}

\begin{Shaded}
\begin{Highlighting}[numbers=left,,]
\FunctionTok{print}\NormalTok{(B)}
\end{Highlighting}
\end{Shaded}

\begin{verbatim}
          [,1]     [,2]      [,3]
[1,] 1.5384615 0.000000 0.0000000
[2,] 0.4487179 2.500000 0.8333333
[3,] 0.9829060 1.666667 2.7777778
\end{verbatim}

\section{\texorpdfstring{Suponha que, devido a mudanças na política
tributária do governo, as demandas finais dos produtos dos setores 1, 2
e 3 sejam projetadas para o próximo ano (ano \(t+1\)) em \(1300\),
\(100\) e \(200\), respectivamente (também medidos em milhares de
reais). Encontre os produtos totais que seriam necessários dos três
setores para atender a essa demanda projetada, assumindo que não há
mudança na estrutura tecnológica da economia (isto é, assumindo que a
matriz \(A\) não muda do ano \(t\) para o ano
\(t+1\))}{Suponha que, devido a mudanças na política tributária do governo, as demandas finais dos produtos dos setores 1, 2 e 3 sejam projetadas para o próximo ano (ano t+1) em 1300, 100 e 200, respectivamente (também medidos em milhares de reais). Encontre os produtos totais que seriam necessários dos três setores para atender a essa demanda projetada, assumindo que não há mudança na estrutura tecnológica da economia (isto é, assumindo que a matriz A não muda do ano t para o ano t+1)}}\label{suponha-que-devido-a-mudanuxe7as-na-poluxedtica-tributuxe1ria-do-governo-as-demandas-finais-dos-produtos-dos-setores-1-2-e-3-sejam-projetadas-para-o-pruxf3ximo-ano-ano-t1-em-1300-100-e-200-respectivamente-tambuxe9m-medidos-em-milhares-de-reais.-encontre-os-produtos-totais-que-seriam-necessuxe1rios-dos-truxeas-setores-para-atender-a-essa-demanda-projetada-assumindo-que-nuxe3o-huxe1-mudanuxe7a-na-estrutura-tecnoluxf3gica-da-economia-isto-uxe9-assumindo-que-a-matriz-a-nuxe3o-muda-do-ano-t-para-o-ano-t1}

\begin{Shaded}
\begin{Highlighting}[numbers=left,,]
\CommentTok{\# novos valores de f}
\NormalTok{f\_new }\OtherTok{=} \FunctionTok{c}\NormalTok{(}\DecValTok{1300}\NormalTok{, }\DecValTok{100}\NormalTok{, }\DecValTok{200}\NormalTok{)}

\CommentTok{\# nova produção final}
\NormalTok{x\_new }\OtherTok{=}\NormalTok{ B }\SpecialCharTok{\%*\%}\NormalTok{ f\_new}

\CommentTok{\# resposta}
\FunctionTok{dimnames}\NormalTok{(x\_new) }\OtherTok{=} \FunctionTok{list}\NormalTok{(}\FunctionTok{c}\NormalTok{(}\StringTok{"setor A"}\NormalTok{, }\StringTok{"setor B"}\NormalTok{, }\StringTok{"setor C"}\NormalTok{), }\FunctionTok{c}\NormalTok{(}\StringTok{"produção total"}\NormalTok{))}
\FunctionTok{print}\NormalTok{(x\_new)}
\end{Highlighting}
\end{Shaded}

\begin{verbatim}
        produção total
setor A           2000
setor B           1000
setor C           2000
\end{verbatim}

\chapter{TERCEIRA QUESTÃO}\label{terceira-questuxe3o}

Considere uma economia organizada em três setores: madeira e produtos de
madeira, papel e produtos afins e maquinário e equipamentos de
transporte. Uma empresa de consultoria estima que no ano passado a
indústria madeireira teve uma produção avaliada em 50 (suponha que todos
os valores monetários estejam em milhões de reais), 5\% dos quais ela
mesma consumiu; 70\% foram consumidos pela demanda final; 20\% pela
indústria de papel e produtos afins; 5\% pela indústria de equipamentos.
A indústria de equipamentos consumia 15\% de seus próprios produtos, de
um total de 100; 25\% foram para a demanda final; 30\% para a indústria
madeireira; 30 por cento para a indústria de papel e produtos afins.
Finalmente, a indústria de papel e produtos afins produzia 50, dos quais
consumia 10\%; 80\% foram para a demanda final; 5\% foram para a
indústria madeireira; e 5\% para a indústria de equipamentos.

\section{Construa a matriz de insumo-produto para esta economia com base
nessas estimativas dos dados do ano passado. Encontre a matriz
correspondente de coeficientes técnicos e mostre que as condições de
Hawkins-Simon são
satisfeitas.}\label{construa-a-matriz-de-insumo-produto-para-esta-economia-com-base-nessas-estimativas-dos-dados-do-ano-passado.-encontre-a-matriz-correspondente-de-coeficientes-tuxe9cnicos-e-mostre-que-as-condiuxe7uxf5es-de-hawkins-simon-suxe3o-satisfeitas.}

\begin{Shaded}
\begin{Highlighting}[numbers=left,,]
\CommentTok{\# vetor de produção}
\NormalTok{x }\OtherTok{=} \FunctionTok{c}\NormalTok{(}\DecValTok{50}\NormalTok{, }\DecValTok{50}\NormalTok{, }\DecValTok{100}\NormalTok{)}

\CommentTok{\# proporções}
\NormalTok{props }\OtherTok{=} \FunctionTok{list}\NormalTok{(}
  \AttributeTok{Z =} \FunctionTok{matrix}\NormalTok{(}
    \FunctionTok{c}\NormalTok{(}
      \FloatTok{0.05}\NormalTok{, }\FloatTok{0.2}\NormalTok{, }\FloatTok{0.05}\NormalTok{,}
      \FloatTok{0.05}\NormalTok{, }\FloatTok{0.10}\NormalTok{, }\FloatTok{0.05}\NormalTok{,}
      \FloatTok{0.3}\NormalTok{, }\FloatTok{0.3}\NormalTok{, }\FloatTok{0.15}
\NormalTok{    ),}
    \AttributeTok{nrow =} \DecValTok{3}\NormalTok{,}
    \AttributeTok{byrow =} \ConstantTok{TRUE}
\NormalTok{  ),}
  \AttributeTok{f =} \FunctionTok{matrix}\NormalTok{(}
    \FunctionTok{c}\NormalTok{(}\FloatTok{0.7}\NormalTok{, }\FloatTok{0.8}\NormalTok{, }\FloatTok{0.25}\NormalTok{),}
    \AttributeTok{nrow =} \DecValTok{3}\NormalTok{,}
    \AttributeTok{byrow =} \ConstantTok{TRUE}
\NormalTok{  )}
\NormalTok{)}

\CommentTok{\# matriz insumo{-}produto}
\NormalTok{M }\OtherTok{=} \FunctionTok{lapply}\NormalTok{(props, }\ControlFlowTok{function}\NormalTok{(matriz) \{}
\NormalTok{  x }\SpecialCharTok{*}\NormalTok{ matriz}
\NormalTok{\})}

\NormalTok{IO }\OtherTok{=} \FunctionTok{do.call}\NormalTok{(cbind, M) }\SpecialCharTok{|\textgreater{}}
\FunctionTok{cbind}\NormalTok{(x)}
\FunctionTok{dimnames}\NormalTok{(IO) }\OtherTok{=} \FunctionTok{list}\NormalTok{(}
  \FunctionTok{c}\NormalTok{(}\StringTok{"setor A"}\NormalTok{, }\StringTok{"setor B"}\NormalTok{, }\StringTok{"setor C"}\NormalTok{),}
  \FunctionTok{c}\NormalTok{(}\StringTok{"setor A"}\NormalTok{, }\StringTok{"setor B"}\NormalTok{, }\StringTok{"setor C"}\NormalTok{, }\StringTok{"demanda final"}\NormalTok{, }\StringTok{"produção total"}\NormalTok{)}
\NormalTok{)}
\FunctionTok{print}\NormalTok{(IO)}
\end{Highlighting}
\end{Shaded}

\begin{verbatim}
        setor A setor B setor C demanda final produção total
setor A     2.5      10     2.5            35             50
setor B     2.5       5     2.5            40             50
setor C    30.0      30    15.0            25            100
\end{verbatim}

\begin{Shaded}
\begin{Highlighting}[numbers=left,,]
\CommentTok{\# matriz de coeficientes técnicos}
\NormalTok{A }\OtherTok{=} \FunctionTok{with}\NormalTok{(M, }\FunctionTok{sweep}\NormalTok{(Z, }\DecValTok{2}\NormalTok{, x, }\AttributeTok{FUN =} \StringTok{"/"}\NormalTok{))}
\FunctionTok{print}\NormalTok{(A)}
\end{Highlighting}
\end{Shaded}

\begin{verbatim}
     [,1] [,2]  [,3]
[1,] 0.05  0.2 0.025
[2,] 0.05  0.1 0.025
[3,] 0.60  0.6 0.150
\end{verbatim}

\begin{Shaded}
\begin{Highlighting}[numbers=left,,]
\CommentTok{\# matriz de Leontief}
\NormalTok{L }\OtherTok{=} \FunctionTok{diag}\NormalTok{(}\DecValTok{3}\NormalTok{) }\SpecialCharTok{{-}}\NormalTok{ A}

\CommentTok{\# condições de Hawkins{-}Simon}
\FunctionTok{all}\NormalTok{(L }\SpecialCharTok{\%*\%}\NormalTok{ x }\SpecialCharTok{\textgreater{}} \DecValTok{0}\NormalTok{)}
\end{Highlighting}
\end{Shaded}

\begin{verbatim}
[1] TRUE
\end{verbatim}

\section{Encontre a inversa de Leontief para esta
economia}\label{encontre-a-inversa-de-leontief-para-esta-economia}

\begin{Shaded}
\begin{Highlighting}[numbers=left,,]
\CommentTok{\# inversa de Leontief}
\NormalTok{B }\OtherTok{=} \FunctionTok{solve}\NormalTok{(L)}
\FunctionTok{print}\NormalTok{(B)}
\end{Highlighting}
\end{Shaded}

\begin{verbatim}
          [,1]      [,2]       [,3]
[1,] 1.0921005 0.2693848 0.04004368
[2,] 0.0837277 1.1539862 0.03640335
[3,] 0.8299964 1.0047324 1.23043320
\end{verbatim}

\section{A recessão da economia neste ano se reflete na queda da demanda
final, conforme tabela a seguir. Qual seria a produção total de todas as
indústrias necessárias para suprir a demanda final reduzida deste ano?
Calcule os vetores de valor agregado e produção intermediária para a
nova tabela de
transações}\label{a-recessuxe3o-da-economia-neste-ano-se-reflete-na-queda-da-demanda-final-conforme-tabela-a-seguir.-qual-seria-a-produuxe7uxe3o-total-de-todas-as-induxfastrias-necessuxe1rias-para-suprir-a-demanda-final-reduzida-deste-ano-calcule-os-vetores-de-valor-agregado-e-produuxe7uxe3o-intermediuxe1ria-para-a-nova-tabela-de-transauxe7uxf5es}

\begin{Shaded}
\begin{Highlighting}[numbers=left,,]
\CommentTok{\# nova demanda final}
\NormalTok{f\_new }\OtherTok{=} \FunctionTok{c}\NormalTok{(}\FloatTok{0.75}\NormalTok{, }\FloatTok{0.90}\NormalTok{, }\FloatTok{0.95}\NormalTok{) }\SpecialCharTok{*}\NormalTok{ M}\SpecialCharTok{$}\NormalTok{f}
\FunctionTok{print}\NormalTok{(f\_new)}
\end{Highlighting}
\end{Shaded}

\begin{verbatim}
      [,1]
[1,] 26.25
[2,] 36.00
[3,] 23.75
\end{verbatim}

\begin{Shaded}
\begin{Highlighting}[numbers=left,,]
\CommentTok{\# novo nível de produção}
\NormalTok{x\_new }\OtherTok{=}\NormalTok{ B }\SpecialCharTok{\%*\%}\NormalTok{ f\_new}
\FunctionTok{print}\NormalTok{(x\_new)}
\end{Highlighting}
\end{Shaded}

\begin{verbatim}
         [,1]
[1,] 39.31653
[2,] 44.60593
[3,] 87.18056
\end{verbatim}

\begin{Shaded}
\begin{Highlighting}[numbers=left,,]
\CommentTok{\# nova matriz de consumo intermediário}
\NormalTok{Z\_new }\OtherTok{=} \FunctionTok{as.vector}\NormalTok{(x\_new) }\SpecialCharTok{*}\NormalTok{ A}
\FunctionTok{print}\NormalTok{(Z\_new)}
\end{Highlighting}
\end{Shaded}

\begin{verbatim}
          [,1]      [,2]       [,3]
[1,]  1.965826  7.863305  0.9829132
[2,]  2.230297  4.460593  1.1151483
[3,] 52.308336 52.308336 13.0770841
\end{verbatim}

\chapter{QUARTA QUESTÃO}\label{quarta-questuxe3o}

Considere uma economia simples de dois setores contendo as indústrias A
e B. A indústria A requer 2 milhões de seu próprio produto e 6 milhões
da produção da indústria B no processo de fornecimento de 20 milhões de
seu próprio produto aos consumidores finais. Da mesma forma, a indústria
B requer 4 milhões de seu próprio produto e 8 milhões de produção da
indústria A no processo de fornecimento de 20 milhões de seu próprio
produto aos consumidores finais.

\section{Construa a tabela de transações de insumo-produto descrevendo a
atividade econômica nesta
economia}\label{construa-a-tabela-de-transauxe7uxf5es-de-insumo-produto-descrevendo-a-atividade-econuxf4mica-nesta-economia}

\begin{Shaded}
\begin{Highlighting}[numbers=left,,]
\CommentTok{\# dados}
\NormalTok{data }\OtherTok{=} \FunctionTok{list}\NormalTok{(}
  \AttributeTok{Z =} \FunctionTok{matrix}\NormalTok{(}
    \FunctionTok{c}\NormalTok{(}
      \DecValTok{2}\NormalTok{, }\DecValTok{6}\NormalTok{,}
      \DecValTok{8}\NormalTok{, }\DecValTok{4}
\NormalTok{    ),}
    \AttributeTok{nrow =} \DecValTok{2}\NormalTok{,}
    \AttributeTok{byrow =} \ConstantTok{TRUE}
\NormalTok{  ),}
  \AttributeTok{f =} \FunctionTok{c}\NormalTok{(}\DecValTok{20}\NormalTok{, }\DecValTok{20}\NormalTok{)}
\NormalTok{)}

\CommentTok{\# matrizes}
\NormalTok{data }\OtherTok{=} \FunctionTok{within}\NormalTok{(data, \{}
\NormalTok{  x }\OtherTok{=} \FunctionTok{rowSums}\NormalTok{(Z) }\SpecialCharTok{+}\NormalTok{ f}
\NormalTok{  A }\OtherTok{=} \FunctionTok{sweep}\NormalTok{(Z, }\DecValTok{2}\NormalTok{, x, }\AttributeTok{FUN =} \StringTok{"/"}\NormalTok{)}
\NormalTok{  L }\OtherTok{=} \FunctionTok{diag}\NormalTok{(}\DecValTok{2}\NormalTok{) }\SpecialCharTok{{-}}\NormalTok{ A}
\NormalTok{  B }\OtherTok{=} \FunctionTok{solve}\NormalTok{(L)}
\NormalTok{\})}

\CommentTok{\# matriz insumo{-}produto}
\NormalTok{IO }\OtherTok{=} \FunctionTok{with}\NormalTok{(data, }\FunctionTok{cbind}\NormalTok{(Z, f, x))}
\FunctionTok{dimnames}\NormalTok{(IO) }\OtherTok{=} \FunctionTok{list}\NormalTok{(}
  \FunctionTok{c}\NormalTok{(}\StringTok{"setor A"}\NormalTok{, }\StringTok{"setor B"}\NormalTok{),}
  \FunctionTok{c}\NormalTok{(}\StringTok{"setor A"}\NormalTok{, }\StringTok{"setor B"}\NormalTok{, }\StringTok{"demanda final"}\NormalTok{, }\StringTok{"produção total"}\NormalTok{)}
\NormalTok{)}
\FunctionTok{print}\NormalTok{(IO)}
\end{Highlighting}
\end{Shaded}

\begin{verbatim}
        setor A setor B demanda final produção total
setor A       2       6            20             28
setor B       8       4            20             32
\end{verbatim}

\section{Encontre a matriz correspondente de coeficientes técnicos e
mostre que as condições de Hawkins-Simon são
satisfeitas}\label{encontre-a-matriz-correspondente-de-coeficientes-tuxe9cnicos-e-mostre-que-as-condiuxe7uxf5es-de-hawkins-simon-suxe3o-satisfeitas}

\begin{Shaded}
\begin{Highlighting}[numbers=left,,]
\CommentTok{\# matriz de coeficientes técnicos}
\FunctionTok{print}\NormalTok{(data}\SpecialCharTok{$}\NormalTok{A)}
\end{Highlighting}
\end{Shaded}

\begin{verbatim}
           [,1]   [,2]
[1,] 0.07142857 0.1875
[2,] 0.28571429 0.1250
\end{verbatim}

\begin{Shaded}
\begin{Highlighting}[numbers=left,,]
\CommentTok{\# condições de Hawkins{-}Simon}
\FunctionTok{all}\NormalTok{(data}\SpecialCharTok{$}\NormalTok{L }\SpecialCharTok{\%*\%}\NormalTok{ data}\SpecialCharTok{$}\NormalTok{x }\SpecialCharTok{\textgreater{}} \DecValTok{0}\NormalTok{)}
\end{Highlighting}
\end{Shaded}

\begin{verbatim}
[1] TRUE
\end{verbatim}

\section{Se no ano seguinte àquele em que foram compilados os dados
desse modelo não fossem esperadas mudanças nos padrões de consumo da
indústria, e se fosse apresentada uma demanda final de 15 milhões do bem
A e 18 milhões do bem B na economia, qual seria a produção total de
todas as indústrias necessárias para suprir essa demanda final, bem como
a atividade interindústria envolvida no suporte às entregas dessa
demanda
final?}\label{se-no-ano-seguinte-uxe0quele-em-que-foram-compilados-os-dados-desse-modelo-nuxe3o-fossem-esperadas-mudanuxe7as-nos-padruxf5es-de-consumo-da-induxfastria-e-se-fosse-apresentada-uma-demanda-final-de-15-milhuxf5es-do-bem-a-e-18-milhuxf5es-do-bem-b-na-economia-qual-seria-a-produuxe7uxe3o-total-de-todas-as-induxfastrias-necessuxe1rias-para-suprir-essa-demanda-final-bem-como-a-atividade-interinduxfastria-envolvida-no-suporte-uxe0s-entregas-dessa-demanda-final}

\begin{Shaded}
\begin{Highlighting}[numbers=left,,]
\CommentTok{\# nova demanda final}
\NormalTok{f\_new }\OtherTok{=} \FunctionTok{c}\NormalTok{(}\DecValTok{15}\NormalTok{, }\DecValTok{18}\NormalTok{)}

\CommentTok{\# nova produção total}
\NormalTok{x\_new }\OtherTok{=}\NormalTok{ data}\SpecialCharTok{$}\NormalTok{B }\SpecialCharTok{\%*\%}\NormalTok{ f\_new}
\FunctionTok{print}\NormalTok{(x\_new)}
\end{Highlighting}
\end{Shaded}

\begin{verbatim}
         [,1]
[1,] 21.74118
[2,] 27.67059
\end{verbatim}

\begin{Shaded}
\begin{Highlighting}[numbers=left,,]
\CommentTok{\# nova matriz de consumo intermediário}
\NormalTok{Z\_new }\OtherTok{=} \FunctionTok{sweep}\NormalTok{(data}\SpecialCharTok{$}\NormalTok{A, }\DecValTok{2}\NormalTok{, x\_new, }\AttributeTok{FUN =} \StringTok{"*"}\NormalTok{)}
\FunctionTok{print}\NormalTok{(Z\_new)}
\end{Highlighting}
\end{Shaded}

\begin{verbatim}
         [,1]     [,2]
[1,] 1.552941 5.188235
[2,] 6.211765 3.458824
\end{verbatim}

\chapter{QUINTA QUESTÃO}\label{quinta-questuxe3o}

Considere as seguintes transações e dados de produção total para uma
economia de oito setores:

\begin{Shaded}
\begin{Highlighting}[numbers=left,,]
\NormalTok{data }\OtherTok{=} \FunctionTok{list}\NormalTok{(}
  \AttributeTok{Z =}\NormalTok{ readxl}\SpecialCharTok{::}\FunctionTok{read\_excel}\NormalTok{(}\StringTok{"data{-}raw/l1\_q5.xlsx"}\NormalTok{, }\AttributeTok{sheet =} \StringTok{"z"}\NormalTok{),}
  \AttributeTok{x =}\NormalTok{ readxl}\SpecialCharTok{::}\FunctionTok{read\_excel}\NormalTok{(}\StringTok{"data{-}raw/l1\_q5.xlsx"}\NormalTok{, }\AttributeTok{sheet =} \StringTok{"x"}\NormalTok{)}
\NormalTok{)}

\FunctionTok{print}\NormalTok{(data)}
\end{Highlighting}
\end{Shaded}

\begin{verbatim}
$Z
# A tibble: 8 x 8
  `setor 1` `setor 2` `setor 3` `setor 4` `setor 5` `setor 6` `setor 7`
      <dbl>     <dbl>     <dbl>     <dbl>     <dbl>     <dbl>     <dbl>
1      8565      8069      8843      3045      1124       276       230
2      1505      6996      6895      3530      3383       365       219
3        98        39         5       429      5694         7       376
4       999      1048       120      9143      4460       228       210
5      4373      4488      8325      2729     29671      1733      5757
6      2150        36       640      1234       165       821        90
7       506         7       180         0      2352         0     18091
8      5315      1895      2993      1071     13941       434      6096
# i 1 more variable: `setor 8` <dbl>

$x
# A tibble: 1 x 8
  `setor 1` `setor 2` `setor 3` `setor 4` `setor 5` `setor 6` `setor 7`
      <dbl>     <dbl>     <dbl>     <dbl>     <dbl>     <dbl>     <dbl>
1     37610     45108     46323     41059    209403     11200     55992
# i 1 more variable: `setor 8` <dbl>
\end{verbatim}

\section{Calcule A e B}\label{calcule-a-e-b}

\begin{Shaded}
\begin{Highlighting}[numbers=left,,]
\CommentTok{\# matriz de coeficientes técnicos}
\NormalTok{A }\OtherTok{=} \FunctionTok{with}\NormalTok{(data, }\FunctionTok{sweep}\NormalTok{(}\FunctionTok{as.matrix}\NormalTok{(Z), }\DecValTok{2}\NormalTok{, }\FunctionTok{as.matrix}\NormalTok{(x), }\AttributeTok{FUN =} \StringTok{"/"}\NormalTok{))}
\FunctionTok{print}\NormalTok{(A)}
\end{Highlighting}
\end{Shaded}

\begin{verbatim}
        setor 1      setor 2      setor 3    setor 4      setor 5    setor 6
[1,] 0.22773199 0.1788817948 0.1908986896 0.07416157 0.0053676404 0.02464286
[2,] 0.04001595 0.1550944400 0.1488461455 0.08597384 0.0161554515 0.03258929
[3,] 0.00260569 0.0008645916 0.0001079377 0.01044838 0.0271915875 0.00062500
[4,] 0.02656208 0.0232331294 0.0025905058 0.22267956 0.0212986442 0.02035714
[5,] 0.11627227 0.0994945464 0.1797163396 0.06646533 0.1416932900 0.15473214
[6,] 0.05716565 0.0007980846 0.0138160309 0.03005431 0.0007879543 0.07330357
[7,] 0.01345387 0.0001551831 0.0038857587 0.00000000 0.0112319308 0.00000000
[8,] 0.14131880 0.0420102864 0.0646115321 0.02608442 0.0665749774 0.03875000
         setor 7    setor 8
[1,] 0.004107730 0.02150498
[2,] 0.003911273 0.01828916
[3,] 0.006715245 0.00203006
[4,] 0.003750536 0.01381931
[5,] 0.102818260 0.09160722
[6,] 0.001607372 0.04170004
[7,] 0.323099729 0.16469558
[8,] 0.108872696 0.28767251
\end{verbatim}

\begin{Shaded}
\begin{Highlighting}[numbers=left,,]
\CommentTok{\# matriz de Leontief}
\NormalTok{L }\OtherTok{=} \FunctionTok{diag}\NormalTok{(}\DecValTok{8}\NormalTok{) }\SpecialCharTok{{-}}\NormalTok{ A}

\CommentTok{\# inversa de Leontief}
\NormalTok{B }\OtherTok{=} \FunctionTok{solve}\NormalTok{(L)}
\FunctionTok{print}\NormalTok{(B)}
\end{Highlighting}
\end{Shaded}

\begin{verbatim}
              [,1]        [,2]       [,3]       [,4]       [,5]        [,6]
setor 1 1.33936754 0.295979543 0.31152758 0.17209767 0.03372393 0.058445918
setor 2 0.08865389 1.213854389 0.20908242 0.15271793 0.03817409 0.057062514
setor 3 0.01289215 0.008946835 1.01106350 0.01945268 0.03398902 0.007975561
setor 4 0.06457251 0.056209000 0.03427157 1.30553565 0.03842939 0.040524907
setor 5 0.26481212 0.215471327 0.31957910 0.17353833 1.20702097 0.230202254
setor 6 0.09986091 0.028837417 0.04541811 0.05887405 0.01106340 1.089050242
setor 7 0.10930070 0.049306100 0.06841558 0.03503482 0.05376392 0.029981515
setor 8 0.32136740 0.162420557 0.20989383 0.11749705 0.13506083 0.102508575
              [,7]       [,8]
setor 1 0.02990929 0.06693474
setor 2 0.02466602 0.05135377
setor 3 0.01750249 0.01276242
setor 4 0.02080916 0.04094366
setor 5 0.22940051 0.23954578
setor 6 0.01776820 0.07431130
setor 7 1.54718657 0.37183110
setor 8 0.26861781 1.50607508
\end{verbatim}

\section{Se as demandas finais nos setores 1 e 2 aumentam em 30\%,
enquanto as do setor 5 diminuem em 20\% (enquanto todas as outras
demandas finais permanecem inalteradas), que novos produtos totais serão
necessários de cada um dos oito setores dessa
economia?}\label{se-as-demandas-finais-nos-setores-1-e-2-aumentam-em-30-enquanto-as-do-setor-5-diminuem-em-20-enquanto-todas-as-outras-demandas-finais-permanecem-inalteradas-que-novos-produtos-totais-seruxe3o-necessuxe1rios-de-cada-um-dos-oito-setores-dessa-economia}

\begin{Shaded}
\begin{Highlighting}[numbers=left,,]
\CommentTok{\# demandas finais atuais}
\NormalTok{f }\OtherTok{=} \FunctionTok{with}\NormalTok{(data, x }\SpecialCharTok{{-}} \FunctionTok{rowSums}\NormalTok{(Z))}

\CommentTok{\# novas demandas finais}
\NormalTok{f\_new }\OtherTok{=}\NormalTok{ f }\SpecialCharTok{*} \FunctionTok{c}\NormalTok{(}\FloatTok{1.3}\NormalTok{, }\FloatTok{1.3}\NormalTok{, }\DecValTok{1}\NormalTok{, }\DecValTok{1}\NormalTok{, }\FloatTok{0.8}\NormalTok{, }\DecValTok{1}\NormalTok{, }\DecValTok{1}\NormalTok{, }\DecValTok{1}\NormalTok{)}

\CommentTok{\# novos produtos totais}
\NormalTok{x\_new }\OtherTok{=}\NormalTok{ B }\SpecialCharTok{\%*\%} \FunctionTok{t}\NormalTok{(}\FunctionTok{as.matrix}\NormalTok{(f\_new))}
\FunctionTok{print}\NormalTok{(x\_new)}
\end{Highlighting}
\end{Shaded}

\begin{verbatim}
             [,1]
setor 1  39997.91
setor 2  51180.82
setor 3  45454.99
setor 4  40403.94
setor 5 177755.66
setor 6  11181.95
setor 7  54928.72
setor 8 158686.88
\end{verbatim}



\end{document}
